%%%% Time-stamp: <2013-02-25 10:31:01 vk>


\chapter*{Abstract}
\label{cha:abstract}


In the last couple of years, Augmented Reality (AR) got a lot of attention, thanks to the development of even smaller but also more powerful computers. They now easily fit into smartphones and head-mounted devices. Because of that and also because of AR games like Pokemon Go, the growth of this technology is greater than ever. In this thesis I describe how I build an application that runs on a head-mounted device, the Microsoft HoloLens. The device gives users instructions on how to build a LEGO model, without the need for using their hands to look it up in a physical booklet. To achieve that, we used a combination of Python to preprocess our input files, and Unity for building our HoloLens application. The final result is a working prototype that shows AR building
instructions and is controlled via voice commands.  



%\glsresetall %% all glossary entries should be used in long form (again)
%% vim:foldmethod=expr
%% vim:fde=getline(v\:lnum)=~'^%%%%\ .\\+'?'>1'\:'='
%%% Local Variables:
%%% mode: latex
%%% mode: auto-fill
%%% mode: flyspell
%%% eval: (ispell-change-dictionary "en_US")
%%% TeX-master: "main"
%%% End:
